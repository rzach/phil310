\chapter{Preface}

Formal logic has many applications both within philosophy and outside
(especially in mathematics, computer science, and linguistics). This
second course will introduce you to the concepts, results, and methods
of formal logic necessary to understand and appreciate these
applications as well as the limitations of formal logic.  It will be 
mathematical in that you will be required to master abstract formal
concepts and to prove theorems \emph{about} logic (not just \emph{in}
logic the way you did in Phil 210); but it does not presuppose any
advanced knowledge of mathematics.

We will begin by studying some basic formal concepts: sets, relations,
and functions and sizes of infinite sets.  We will then consider the
language, semantics, and proof theory of first-order logic (FOL), and ways
in which we can use first-order logic to formalize facts and reasoning
abouts some domains of interest to philosophers and logicians.

In the second part of the course, we will begin to investigate the
meta-theory of first-order logic.  We will concentrate on a few
central results: the completeness theorem, which relates the proof
theory and semantics of first-order logic, and the compactness theorem
and L\"owenheim-Skolem theorems, which concern the existence and size
of first-order interpretations.

In the third part of the course, we will discuss a particular way of
making precise what it means for a function to be computable, namely,
when it is recursive.  This will enable us to prove important results
in the metatheory of logic and of formal systems formulated in
first-order logic: G\"odel's incompleteness theorem, the Church-Turing
undecidability theorem, and Tarski's theorem about the undefinability
of truth.
 
\begin{description}
\item[Week 1] (Jan 5, 7). Introduction. Sets and Relations.

\item[Week 2] (Jan 12, 14). Functions. Enumerability.

\item[Week 3] (Jan 19, 21). Syntax and Semantics of FOL.

\item[Week 4] (Jan 26, 28). Structures and Theories. 

\item[Week 5] (Feb 2, 5). Sequent Calculus and Proofs in FOL.

\item[Week 6] (Feb 9, 12). The Completeness Theorem.

\item[Week 7] (Feb 16, 18). Compactness and L\"owenheim-Skolem Theorems

\item[Week 8] (Mar 23, 25). Recursive Functions 

\item[Week 9] (Mar 9, 11). Arithmetization of Syntax

\item[Week 10] (Mar 16, 18). Theories and Computability

\item[Week 11] (Mar 23, 25). G\"odel's Incompleteness Theorems

\item[Week 12] (Mar 30, Apr 1). The Undefinability of Truth. 

\item[Week 13, 14] (Apr 8, 13). Applications.
\end{description}

